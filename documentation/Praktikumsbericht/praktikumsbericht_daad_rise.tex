\documentclass{article}
\usepackage{a4}                % Paket fuer A4 Seitenformat
\usepackage{epsfig}            % Paket zum Einbinden von postscript Graphiken
\usepackage[ngerman]{babel}    % deutsche Sonderzeichen, Trennmuster, etc
\usepackage[latin1]{inputenc}  % deutsche Umlaute
%\usepackage{float}            % Hilfesmakros beim Positionieren von Graphiken

%%%%%%%%%%%%%%%%%%%%%%%%%%%%%%%%%%%%%%%%%%%%%%%%%%%%%%%%%%%%%%%%%%%%%%%%%%%%%%%%%%%%%%%%%%%%%%%%%
%%  ein paar kleine Modifikationen am Format
%%%%%%%%%%%%%%%%%%%%%%%%%%%%%%%%%%%%%%%%%%%%%%%%%%%%%%%%%%%%%%%%%%%%%%%%%%%%%%%%%%%%%%%%%%%%%%%%%

%% Zeilenabstand: 1 1/2-zeilig   (1.3)
\renewcommand{\baselinestretch}{1.3}

%% Absatzabstand
\parskip1.5ex

%% erste Zeile eines Absatzes nicht einrcken
\parindent0em

%% Seitenumbruchsteuerung, bei wenigen Zeilen nach �erschrift am Zeilenende
\def\condbreak#1{\vskip 0pt plus #1\pagebreak[3]\vskip 0pt plus -#1\relax}

\title{Abschlussbericht \"uber das Forschungspraktikum an der Nottingham Trent University im Rahmen des DAAD Rise Programmes}
\author{Hannes Thielke}

%% Beginn

\begin{document}

\maketitle         % erstellt Dokument-Kopf mit Titel und Authoren wie oben definiert

%\tableofcontents   % erstellt Inhaltsverzeichnis (bei Bedarf auskommentieren)

\newpage                % neue Seite

\section{Einleitung}
Ein Praktikum \"uber das DAAD Rise Programm ist die perfekte Gelegenheit, die eigenen fachlichen Kompetenzen auch in der Praxis auszubauen und gleichzeitig wertvolle Eindr\"ucke aus anderen L\"andern zu bekommen. 
Ich habe mich f\"ur England entschieden um meine Sprachkentnisse auszubauen, was mir nach meiner pers\"onlichen Einsch\"atzung auch gelungen ist. Der gr\"osste Forschritt ist allerdings wie ich finde, dass die anf\"angliche Angst davor, Englisch sprechen, relativ schnell verfliegt und weniger dar\"uber nachdenken muss, was man sagen m\"ochte.
Ich empfehle es jedem, der dar\"uber nachdenkt, sich f\"ur dieses Programm zu bewerben. Es lohnt sich!

\section{Vorbereitung auf das Praktikum}
Ich habe mich gegen Ende des letzten Jahres auf das Rise Praktikum beworben.  Es macht auf jeden Fall Sinn, sich mehr Zeit daf\"ur einzuplanen, da der Bewerbungsprozess relativ langwierig sein kann. Man kann sich insgesamt auf drei verschiedene Stellen bewerben und sollte daf\"ur jeweils eine eigene Bewerbung schreiben. Man sollte nicht die gleiche Bewerbung f\"ur alle drei Stellen zu verwenden.
Es ist ausserdem ein Empfehlungsschreiben eines Hochschullehrers einzureichen. Dieser muss mindestens einen Doktor-Titel besitzen. Falls man nicht gerade pers\"onlich mit jemandem in Kontakt steht, empfiehlt es sich, in einer Vorlesung pers\"onlich mit dem Hochschullehrer zu sprechen und demjenigen einige Zeit als Vorlauf zu geben. Oft dauert soetwas mehr als nur ein paar Tage.
Nach dem Bewerbungs- und Auswahlverfahren bekommt man (hoffentlich) vom DAAD eine Zusage f\"ur das Praktikum und kann sich nun voll auf die Reiseplanung konzentrieren.

\section{Anreise}
Zur Zeit meines Praktikums war England noch Mitglied der EU und somit relativ einfach zu erreichen. Da sich Fl\"uge nach Nottingham als relativ teuer herausgestellt haben, habe ich mich daf\"ur entschieden, nach London zu fliegen und von dort aus mit anderen Verkehrsmitteln weiter zu fahren. Nach meinen Erfahrungen sind die Fluh\"afen in London Stansted und in Manchester am billigsten zu erreichen. Von dort aus fahren viele Busse und Z\"uge in alle gr\"osseren St\"adte. Falls man vorhat, viel in der UK unterwegs zu sein, empfielt sich der Kauf einer Railcard f\"ur Busse von Nationalrail oder eine Couchcard von Nationalexpress. Ein weiterer Billig-Anbieter ist Megabus. Dabei sollten die Busse bereits im voraus gebucht werden, da die Pl\"atze gegen Ende hin teurer werden.

\section{Wohnungssuche}
Die Wohnungssuche ist eine der nervenzerreibendsten Schritte bei der Planung. Oft werden Wohnungen erst relativ sp\"at - teilweise erst einige Wochen vorher - ausgeschrieben, was es schwierig macht, eine Wohnung bereits viel im Voraus zu bekommen. Einzelwohnungen und WGs in England k\"onnen \"uber die Wohnungsb\"orse spareroom.co.uk gefunden werden. Es lohnt sich, hier einen Account zu erstellen und regelm\"assig (d.h. t\"aglich) neue Angebote zu pr\"ufen, da die Wohnungen relativ schnell vergriffen sein k\"onnen. Ausserdem hilft es auch, eine "Wohnung gesucht"-Anzeige zu schalten, in der man ein wenig \"uber sich selbst und den Grund seine Wohnungssuche schreibt. So k\"onnen Vermieter direkt mit dir in Kontakt treten. So habe z.B. ich meine Wohnung bekommen. Allerdings sollte man immer darauf Acht geben, ob es sich dabei um eine Agentur oder einen Makler handelt, da diese h\"aufig Vermittlungsgeb\"uhren verlangen.

\end{document}   % hier ist das Ende des Dokuments